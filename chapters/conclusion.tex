\chapter*{结语} 
\markboth{结语}{}  
\phantomsection 
\addcontentsline{toc}{chapter}{结语} 

在给出了在引入节点最大度限制之后,利用分治和递归的思想,对无尺度网络
进行多层构建,对所构造的网络进行度-度相关性,以及聚类性分析。

\begin{table}
  \centering
  \begin{tabular}{cccp{38mm}}
    \toprule
    \textbf{文档域类型} & \textbf{Java类型} & \textbf{宽度(字节)} & \textbf{说明} \\
    \midrule
    BOOLEAN  & boolean &  1  & \\
    CHAR     & char    &  2  & UTF-16字符 \\
    BYTE     & byte    &  1  & 有符号8位整数 \\
    SHORT    & short   &  2  & 有符号16位整数 \\
    INT      & int     &  4  & 有符号32位整数 \\
    LONG     & long    &  8  & 有符号64位整数 \\
    STRING   & String  &  字符串长度  & 以UTF-8编码存储 \\
    DATE     & java.util.Date & 8 & 距离GMT时间1970年1月1日0点0分0秒的毫秒数 \\
    BYTE\_ARRAY & byte$[]$ & 数组长度 & 用于存储二进制值 \\
    BIG\_INTEGER & java.math.BigInteger & 和具体值有关 & 任意精度的长整数 \\
    BIG\_DECIMAL & java.math.BigDecimal & 和具体值有关 & 任意精度的十进制实数 \\
    \bottomrule
  \end{tabular}
  \caption{测试表格}\label{table:test5}
\end{table}

用于测试表格。随后分析了无尺度网络构造过程中,交换机节点与数
据节点的角色区别,分析了两者在不同比率下形成的网络形态,以及对网络性能造成的影响。

通过理论分析和仿真实验,分析并找出比率因子q的最佳取值。此外,无尺度现象
的引入提高了网络的聚类系数,从而在不失灵活性可靠性的基础上,进一步提升
了网络的性能。

将关注点转移到交换机\index{交换机}本身。由于图论难以描述数据中心
网络中的交换设备,因此放弃基于图的抽象模型,转而基于多维簇划分的思想,提出并设计
了WarpNet网络模型。

该网络模型突破了基于图描述的局限性,并对网络的带宽等指标进行理论分析并
给出定量描述。最后对比了理论分析、仿真测试结果,并在实际物理环境中进系
真实部署,通过6节点的小规模实验以及1000节点虚拟机的大规模实验,表明该模
型的理论分析、仿真测试与实际实验吻合,并在网络性能、容错能力、伸缩性灵
活性方面得到了进一步的提升。

在第\ref{chapter_introduction}章中,针对网络模型研究这一类工作的共性,设计构造通
用验证平台系统。以海量虚拟机和虚拟分布式交换机的形式,实现了基于少量物理节点,对
大规模节点的模拟。其模拟运行的过程与真实运行在实现层面完全一致,运行的结果与真实
环境线性相关。除为本文所涉若干网络模型提供验证外,可进一步推广到更为广泛的领域,
为各种网络模型及路由算法的研究工作,提供分析、指导与验证。